


\chapter{Coding rules}


\section{Automatic formatting with \Astyle}

\subsection{Usage of \Astyle}

Automatic formatting of C++ files may be performed by means of a command-line tool named \Astyle, also known as \texttt{Artistic Style}. The official website of this tool is 
\begin{center}
\url{http://astyle.sourceforge.net}
\end{center}
Documentation about this tool and how it could be used is available on this website. For the sake of simplicity, we only emphasize on how \Astyle is used in \Simol. On Ubuntu systems, \Astyle can be installed from the command line
\begin{minted}{bash}
sudo apt-get install astyle
\end{minted}
We use the following command line from the root directory of \Simol to C++ source file be conforming to our coding rules:
\begin{minted}{bash}
astyle --recursive --style=allman --indent=spaces=2 --convert-tabs \
       --indent-namespaces --indent-classes --indent-col1-comments \
       --indent-switches --indent-labels --keep-one-line-blocks \
       --pad-oper --suffix=none *.cpp
\end{minted}
To do the same for header files, we use the command line below:
\begin{minted}{bash}
astyle --recursive --style=allman --indent=spaces=2 --convert-tabs \
       --indent-namespaces --indent-classes --indent-col1-comments \
       --indent-switches --indent-labels --keep-one-line-blocks \
       --pad-oper --suffix=none *.hpp
\end{minted}
To understand the meaning of this command line, please refer to the official documentation.


\section{Preprocessor directives}


\subsection{Avoid using \texttt{pragma once}}

The directive named \texttt{pragma once} does not belong to C++ standard. Prefer using include guards in header files instead. For example, the following \texttt{foo.hpp} file is correct
\begin{cppcode}
	#ifndef SIMOL_FOO_HPP
	#define SIMOL_FOO_HPP
	class Foo;
	#endif
\end{cppcode}
but the version below is not officially supported:
\begin{cppcode}
	#pragma once
	class Foo;
\end{cppcode}
Note that if \texttt{pragma once} is added to the standard someday, then it should be preferred to include guards. But it is not planned to be added to it for the moment.


\section{Namespaces}


\subsection{Avoid using \textit{using namespace}}

Do not use the instruction \textit{using namespace} because it cancels the effects of namespaces and may create name conflicts. For example, such a conflict may appear in the code below if a type named \texttt{string} is defined in \texttt{string.hpp}.

\begin{cppcode}
	#include "string.hpp"
	using namespace std;
	int main()
	{
		string hello = "hello"; // std::string or string from string.hpp?
		string world = "world"; // std::string or string from string.hpp?
		cout << hello << endl;
		cout << world << endl;
	}
\end{cppcode}

Thus, please prefer avoiding using the \texttt{using namespace foo} instruction. Use \texttt{foo::bar} when you want to use \texttt{bar} function defined in \texttt{foo} namespace. According to this rule, the correct version of the code below is
\begin{cppcode}
	#include "string.hpp"
	int main()
	{
		std::string hello = "hello"; // ok, std::string
		string world = "world"; // ok, string from string.hpp
		std::cout << hello << std::endl;
		std::cout << world << std::endl;
	}
\end{cppcode}

\subsection{Avoid using \textit{using foo::function}} 

For the same reason as above, do not use the instruction \textit{using namespace::function}.




\chapter{Development environment}

\section{File organization}
\section{Build system}

\subsection{Why \CMake?}

The build system of \Simol is \CMake, which is very similar to \Make. A typical installation of a software with \CMake consists in running the three following commands from a subdirectory usually called \url{build}: 
\lstset{language=bash} 
\begin{lstlisting}
cmake src && make && make install
\end{lstlisting}
If you have already installed a software with \Make, you probably noticed the similarity with the classical combo:
\lstset{language=bash} 
\begin{lstlisting}
src/configure && make && make install
\end{lstlisting}
The only difference between \Make and \CMake is the first command, which corresponds to the configuration step. This is one of the reason why \CMake has been chosen to automatize the installation of \Simol. The second reason is that \CMake is a cross-platform build system. Contrary to \Make, which is only available on Unix systems, \CMake can be used on Windows systems too.

\subsection{How does it work?}

\CMake relies on a hierarchy of configuration files named \CMakeLists. The main configuration file must be located at the top directory of the project.







\section{Version control wit \Git}


\subsection{Main commands}


\subsubsection{Clone}

\subsubsection{Add}

\subsubsection{Commit}


\subsubsection{Push}


\subsubsection{Status}


\subsubsection{Pull}


\subsection{Good practices}


\subsubsection{Atomic commits}

\subsubsection{Writing good commit messages}



\section{Coding rules}
\section{Unit tests}
\section{Continuous integration}
\section{Documentation}